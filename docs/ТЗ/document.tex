\documentclass[a4paper, 10pt]{article}

%\usepackage{cmap}
\usepackage[T2A]{fontenc}
\usepackage[utf8]{inputenc}
\usepackage[english, russian]{babel}
\usepackage{graphicx}
\usepackage[top=2cm, bottom=2cm, left=3cm, right=2cm]{geometry}
\graphicspath{./}
\usepackage{biblatex}
\addbibresource{lib.bib}
\linespread{1.5}
{\usefont{T2A}{Times New Roman}{m}{n}
\usepackage{amsmath}

\usepackage{ragged2e}
\justifying

\usepackage{listings}
\usepackage{color}


\begin{document}
	
\begin{titlepage}
	\fontsize{14pt}{14pt}\selectfont

	\begin{center}
		\centering{\hspace{10mm} \small \bf Министерство науки и высшего образования Российской Федерации
			Федеральное государственное бюджетное образовательное учреждение
			высшего образования \\ <<Московский государственный технический университет имени Н.Э.Баумана\\(национальный исследовательский университет)>>\\(МГТУ им. Н.Э.Баумана)} 
		\noindent\rule{\textwidth}{2pt}
	\end{center}
	\begin{flushright}
		\normalsize{УТВЕРЖДАЮ \\
		Заведующий кафедрой$\underset{\text{(Индекс)}}{\underline{\text{ИУ7}}}$ 
		\\ \vspace{1mm} $\underset{}{\underline{\hspace{3cm}}}$ \hspace{2mm}$\underset{\text{(И.О.Фамилия)}}{\underline{\text{И.В.Рудаков}}}$
		\\ \vspace{1mm}<<$\underset{}{\underline{\hspace{0.7cm}}}$ >> $\underset{}{\underline{\hspace{3cm}}}$2021 г.} 
	\end{flushright}

	\begin{center}
		\large{\bf{ЗАДАНИЕ
		\\ на выполнение курсовой работы}}
	\end{center}
	\begin{flushleft}
		\normalsize{по дисциплене $\underset{}{\underline{\hspace*{1cm} \text{Операционные системы} \hspace*{87mm}}}$
		\\Студент группы \underline{\hspace{1cm} \text{ИУ7-76Б} \hspace{1cm}}
		\\ $\underset{ \text{(Фамилия, имя, отчество)}}{\underline{\hspace*{6cm} \text{Турсунов Жасурбек Рустамович} \hspace*{47mm}}}$
		\\Тема курсовой работы \underline{\text{\hspace{1mm} Реализация загружаемого модуля ядра для отслеживания USB-устройств,}}
		\\ \underline{их идентификации, предоставления или отказа в доступе. \hspace*{65mm}}
		\\ Направленность КР (учебная, исследовательская, практическая, производственная, др.)
		\\ \underline{\hspace{6cm} \text{учебная} \hspace{85mm}}
		\\ Источник тематики (кафедра, предприятие, НИР)\underline{\hspace{2cm} \text{кафедра} \hspace{42mm}}
		\\График выполнения проекта:  25\% к \underline{\hspace*{0.5cm}} нед., 50\% к \underline{\hspace*{0.5cm}} нед., 75\% к \underline{\hspace*{0.5cm}} нед., 100\% к \underline{\hspace*{0.5cm}} нед.}
	\end{flushleft}
	\normalsize {{ \textbf{\textit{Задание}}} \underline{Необходимо реализовать загружаемый модуль ядра для отслеживания изменений в USB-\hspace*{1mm}} \\ \underline{портах и проверки на наличие доступа к секретным файлам. В случае отстутствия допуска - зашиф- } \\ \underline{ровать все запрещенные для копирования файлы.\hspace*{8cm}}}
	\\ \normalsize {{\textbf{\textit{Оформление курсовой работы:}}}}
	\\ Расчетно-пояснительная записка на \underline{20-30} листах формата А4.
	
	\underline{Расчетно-пояснительная записка должна содержать введение, аналитическую часть, конструк-} \\ \underline{торскую часть, технологическую часть, экспериментально-исследовательский раздел, заключение,} \\ \underline{список литературы, приложения. \hspace*{10.5cm}}
		
	\begin{flushleft}
		\small Дата выдачи задания <<\underline{\hspace{1cm}}>> \underline{\hspace{3cm}} 2021 г.
		\newline
		\\ \small \textbf{Руководитель курсового проекта}
		\small \hspace{3cm}$\underset{\text{(Подипсь, дата)}}{\underline{\hspace{4cm}}}$ 
		\small \hspace{4mm}$\underset{\text{(И.О.Фамилия)}}{\underline{\text{Н.Ю.Рязанова}}}$ 
	\end{flushleft}
	\begin{flushleft}
		\small \textbf{Студент}
		\small \hspace{7.2cm}$\underset{\text{(Подипсь, дата)}}{\underline{\hspace{4cm}}}$ 
		\small \hspace{5mm}$\underset{\text{(И.О.Фамилия)}}{\underline{\text{Ж.Р.Турсунов}}}$ 
	\end{flushleft}
	
\end{titlepage}

\end{document}